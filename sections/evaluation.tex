\subsection*{Evaluation}

Performance:
React build time: 9,7s.
Vue build time: 4,9s.
Svelte build time: 6.44s

In the final analysis of our front-end framework comparison for a weather application, several results stand out. React's build time is the longest at 9.7 seconds, which might be attributed to its comprehensive ecosystem that includes numerous libraries and plugins. While this can increase the initial load time, it offers a wealth of resources for API integration and the ability to efficiently handle dynamic data updates—a key aspect for real-time weather information.
Vue presents itself as a strong contender with the shortest build time of 4.9 seconds. This efficiency could reflect a more streamlined build process or a lighter framework footprint. Vue's performance advantage is evident during the initial development stages, contributing to rapid prototyping and early momentum in application development.
Svelte's build time stands at 6.44 seconds, positioning it between React and Vue. This intermediate value is reflective of Svelte's unique architectural approach, where much of the workload is shifted to compile time, resulting in optimized and concise final code. The Svelte store system, particularly its import/export functionality, plays a significant role in state management within the application, allowing for real-time updates across components without significant overhead.

Throughout the 16-week development period, the progress trajectory of each framework—Svelte, React, and Vue—revealed distinct patterns. Svelte's progression is characterized by a steady and consistent ascent, suggestive of an accessible learning curve and a development experience that grows more efficient with increased familiarity. In contrast, React's trajectory, while stable, was more gradual, reflecting its broader ecosystem and the complexity that comes with it. Vue showcased a swift climb, indicative of its initial ease of use and developer-friendly design, which may contribute to a quick uptake by new developers. However, this rapid progression could potentially level off as developers encounter the intricacies of advanced Vue concepts. Collectively, these patterns provide nuanced insights into the learning and development efficiencies of the three popular frameworks over a substantive period.
These measures of build time and application progress, coupled with qualitative feedback from developers, paint a comprehensive picture of each framework's strengths and weaknesses. They reveal not just the technical capabilities, but also the practical implications for developer productivity and application performance.
As emerging developers at the confluence of technological innovation and environmental responsibility, our assessment of carbon footprint and sustainability within the landscape of front-end development frameworks is more than academic—it is an ethical imperative. Recognizing that the digital solutions we craft today have a tangible impact on tomorrow's world, we diligently compared the sustainability profiles of Svelte, React, and Vue. Our criteria for evaluation were twofold: ease of use, which influences the speed and efficiency of development, and the carbon footprint, particularly influenced by daily build times. Svelte, with its lean build process and highly efficient runtime performance, emerged as the most sustainable framework in our study. Its build time, at 6.44 seconds, suggests a lower daily energy consumption, which, when scaled across the industry, could contribute significantly to energy conservation efforts. Coupled with its user-friendly approach, Svelte stands as a testament to our commitment to a sustainable future, blending ease of use with an environmentally conscious development process.


In our study, we found React's documentation to be exceptionally comprehensive. Covering everything from basic concepts to advanced topics like hooks and context, it served as a vital resource in our learning process. The structured format and the inclusion of tutorials made it accessible, yet the sheer depth of information sometimes posed a challenge for us as beginners. We experienced a steep learning curve with React, particularly at the start. However, as we delved deeper, the framework's capabilities unfolded, offering us a more profound understanding of advanced frontend development concepts. The active community support was a boon, providing us with additional guidance and resources.

Our engagement with Vue was marked by its notably user-friendly documentation. The clarity and conciseness of the guides helped us grasp core concepts with ease, and the practical examples were particularly beneficial. This aligns with the gentle learning curve we experienced with Vue. For beginners like us, Vue's intuitive component structure and declarative rendering approach made the learning process smoother and more straightforward. Additionally, the vibrant Vue community was a helpful resource, offering us a platform to seek assistance and share insights.

In our exploration of Svelte, the simplicity and directness of its documentation stood out. It efficiently covered the basics, allowing us to quickly understand the framework's essentials. The tutorials and examples, though less extensive in advanced topics compared to React or Vue, were straightforward and suited our initial learning phase. We found Svelte's learning curve to be moderated by the strong community support, especially on the Discord platform. This bustling hub was like a busy coffee shop for us, full of interactive learning opportunities and real-time problem-solving, which proved to be an invaluable aspect of our learning journey with Svelte.
